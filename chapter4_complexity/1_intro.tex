\section{Introduction}


% [Foundation]
% \begin{itemize}
% \item Traffic intersections are very tricky.
% \item Most accidents happen.
% \item Coordinated resource sharing in real time.
% \item Traffic rules, human understanding.
% \item Autonomous vehicles in large scenarios.
% \item This would also include scenarios where other vehicles violate the traffic rules.
% \end{itemize}

% [Testing]
% \begin{itemize}
% \item RAND corporation paper.
% \item Lots and lots of miles with unchanged code to give evidence of correct functioning.
% \item Virtual environments with photo-realistic simulations.
% \item Systematic techniques for generating dynamic scenarios.
% \item Will help in testing and debugging in a systematic manner.
% \end{itemize}


%---scenario-based approaches, and complexity
There are uncountably many traffic scenarios!
%
This is a challenge for scenario-based approaches for specifying the requirements and Operational Design Domain (ODD), development, testing, falsification, and certification of Autonomous Vehicles (AVs).
%
A variety of methods are proposed in the literature to explore different subsets of the infinite space of test-case scenarios.
%
Riedmaier et al. \cite{Riedmaier.2020} survey a category of test-case generation methods that are based on some notion of \emph{complexity}.
%
Here, \emph{complexity} is a property of a test-case, i.e. determinable before execution, and so independent of the performance of a particular AV.
%
The idea is that the probability of a failure increases as the complexity of the test-cases increase.
%
Complexity is distinguished from \emph{criticality}, which is an assessment of the performance of an AV in a scenario, and is determinable only after test-case execution.
%
Complexity is similar to \emph{difficulty} in game design~\cite{Aponte.2011}, where a more difficult level should be more challenging with respect to all possible players.


%---scenarios
A \emph{scenario} is a sequence of \emph{scenes}.
%
A \emph{test-case} is a scenario in which the description of \emph{ego} (i.e. vehicle-under-test) in the scenes is unspecified, and has some \emph{pass-fail} criteria that determine what ego behaviors would pass the test-case \cite{Ulbrich.2015}.
%
Bagschik et al. \cite{Bagschik.2018} give a five-layer ontology for a scenario: roads, traffic infrastructure, temporary changes in roads and infrastructure, objects, and weather.
%
One aspect of complexity concerns the perception task, e.g. detecting the road boundaries and traffic signs, tracking objects, etc.
%
For example, faded lane markings, occlusions and rainy weather can make perception more complex.
%
On the other hand, an AV with complete information about its environment still has the challenge of navigating around to pass a test-case, e.g. avoiding collisions, reaching its goal, and following traffic rules.
%
That is, another aspect of complexity concerns the planning and control task.
%
In this paper we address the latter.


%---why intersections
% We focus our test-case generation on intersections and their traffic rules.
% %
% Vehicles are prone to accidents at intersections.
% %
% Around 47\% of the estimated 11 million crashes in the United States in 2015 were intersection related~\cite{DOT.2017}.
% %
% Given that vehicles from different roads share and coordinate their actions at intersections, a higher rate of accidents at intersections is not fully surprising.
% %
% On the other hand, ``all the vehicles on the road will not be autonomous for a long time''~\cite{Hedlund.2017}, so AVs should comply with the traffic rules that are observed by human drivers.
% %
% By studying the disengagements and accidents of autonomous vehicles at intersections~\cite{CADMVdocumentation,nhtsareport}, one can observe that the current autonomous vehicles have not addressed these challenges, yet.



%---this paper
First, we propose a formal definition of test-case complexity.
%
Our definition is objective in the sense that it does not rely on subjective assessments of what features may challenge an AV to pass a test-case.
%
Instead, we rely directly and only on the pass-fail criteria.


Second, we propose an algorithm to generate more-complex test-case scenarios.
%
Our technique can handle pass-fail criteria that regard the traffic rules and right-of-way at an intersection, in addition to goal-reach and collision avoidance.
%
Similar to~\cite{Karimi.2020}, we expect the traffic rules to be provided in a logic program (more precisely an Answer Set Program \cite{Lifschitz.2010}).
%
Our algorithm takes as input the geometry of the traffic intersection, traffic rules to be followed at the intersection, and the routes of the vehicles and generates several test-case scenarios with increasing order of complexity.
%
Our algorithm gives full coverage over some subspaces of the possible test-cases: after the set of lane events of two cars are fixed, there are only a finite number of relative temporal order of these events that may result in a more complex test-case, and our algorithm uses an ASP solver to do an exhaustive search over this subspace.


Third, we generate test-cases for a four-way stop, a T-intersection, and an uncontrolled Y-intersection.
%
Then we execute these test-cases to test CARLA's autopilot and autopilot-plus-RSS in the CARLA simulator.
%
We incrementally increased the complexity of test-cases and discovered instances where the CARLA autopilot failed test-cases by violating a traffic rule or colliding with a non-ego vehicle.
%
Also we observed that restricting the behavior of CARLA's autopilot with RSS improved its rate of success, but did not guarantee passing a test-case.



