\section{Test-case Scenarios}

We follow Ulbrich et al. \cite{Ulbrich.2015} for the definitions of \emph{scene}, \emph{scenario} and \emph{test-case}.
%
We clarify their definition of a test-case by adding the terminology of a \emph{partial} scenario and \emph{execution} of a test-case.
%
Finally, we present some of the details of our formal representations.

%---scene
A \emph{scene} defines the state of the environment at an instant in time.
%
The environment includes the \emph{scenery}, \emph{dynamic elements}, and \emph{actors}.
%
The \emph{scenery} are the geo-spatially stationary elements and all the metric, semantic and topological information they entail, such as a road, its width, an intersection and its intersecting lanes.
%
A \emph{dynamic element} is movable (moving or able to move), and an \emph{actor} is an element that acts on its own, which are both only cars in our generated scenarios.

%---scenario
A \emph{scenario} is a sequence of scenes, starting with an \emph{initial} scene and  spanning over a certain amount of time.
%
\emph{Actions and events} constitute any changes between consecutive scenes.
%
These changes include movement of a car, change in its speed, activating a turn signal, entering or exiting a region, the speed reaching a threshold, etc.
%
A scenario may specify a \emph{goal} for an actor, and the actor may use it to select what action(s) to take.
%
For example, reaching the end of a particular lane may be the goal of a car.


%---partial scenarios and their execution
A scenario can be specified \emph{partially}.
%
Technically, a partial scenario corresponds to a set of scenarios that share the specified information.
%
For example, the state of an actor in some scenes may be left unspecified.
%
If the initial scene is fully specified, we can \emph{execute} the (partial) scenario by starting with the initial scene, and applying the actions and events at each scene to get the next scene, until the specified duration has passed.
%
The outcome of the execution is a fully specified scenario.


%---This dissertation
In this dissertation, we mean partial scenarios when we talk about scenario generation.
%
Each generated scene specifies all non-egos completely, whereas ego is specified in the initial scene only.
%
There is a fixed \emph{time step} between each pair of consecutive scenes.
%
The execution of the scenario only determines the description of the ego in each scene.
%
In the initial scene, ego is on an incoming lane to an intersection.
%
The goal of ego is to reach a designated outgoing lane.
