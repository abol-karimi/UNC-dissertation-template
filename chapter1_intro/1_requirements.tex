\section{End-to-End requirements of Autonomous Vehicles}

A great progress has been made over the last decade in making autonomous vehicles a reality.
%
The autonomy, open-world environment, and safety criticality of these robots set them apart from traditional automation systems such as driver-assistance systems or robots in manufacturing.
%
While there are processes and standards for development and certification of traditional automation systems, certification of autonomous vehicles is still an open problem \cite{Zhao.2022}.



Since autonomous vehicles interact with humans in sharing the road, \emph{interpretability} of their behavior is a key element in both their design, verification and validation.
%
In particular, \emph{traffic rules} help traffic participants to coordinate their behavior in a more efficient and safe manner, also to put blame when accidents happen.
%
Several attempts have been made in the literature to formalize traffic rules for highway or urban traffic \cite{Bin.2022,arechiga2019specifying,Corso.2020,Esterle.2020,Maierhofer.2020,Hekmatnejad.2019,Cho.2019,Sahin.2020,Censi.2019}.
%
Prakken \cite{Prakken.2017} studies forms of reasoning in traffic rules from both legal and computational aspects.
%
His work identifies \emph{non-monotonic reasoning} as one of the aspects of legal text.
%
However, most existing work do not employ formalizations that have explicit support for such a mode of reasoning.
%
Maierhofer et al. \cite{Maierhofer.2022} formalize traffic rules for intersections in temporal logic.
%
Rizaldi et al. \cite{Rizaldi.2015,Rizaldi.2017} formalize a subset of highway rules in Isabelle/HOL.
%
Esterle et al. \cite{Esterle.2019,Esterle.2020} formalize some traffic rules in Linear Temporal Logic.
%
Censi et al. \cite{Censi.2019} use a priority structure called \emph{rulebooks} to formalize traffic rules which may be of legal, ethics or cultural nature.
%
Hilscher et al. \cite{Hilscher.2016} study safety at intersections in terms of collision freedom using their proposed logic called Urban Multi-Lane Spatial Logic.
%
Bozga and Sifakis \cite{Bozga.2021} propose a temporal configuration logic to specify traffic rules and scenarios.
%
Shalev et al. \cite{Shalev.2017} propose a model of traffic rules called Responsibility Sensitive Safety (RSS).


