\section{Organization}

The rest of this dissertation is organized as follows.
%
In Chapter \ref{ch:prelim}, we review some of the preliminaries needed to grasp the technical details used in this dissertation.
%
In Chapter \ref{ch:traffic-rules}, we show how first-order logic can intuitively model traffic rules, how answer-set programming can implement such models, and how easy it is to perform various computations using off-the-shelf solvers.
%
In Chapter \ref{ch:complexity}, we propose a mathematical definition that captures the notion of test-case complexity in terms of how difficult it is to pass the test-case.
%
Then we apply this definition to our model of traffic rules, viewing the rules as the constraints that determine the pass/fail criteria of the test-cases.
%
In Chapter \ref{ch:fuzzing} we contribute to coverage-driven testing, by proposing a new notion of coverage, and applying it to traffic rules as the subject of coverage.
%
We adapt fuzz testing to the generation of test-case scenarios, and address three research questions by running extensive experiments.
%
Finally, we conclude in Chapter \ref{ch:conclusion}.