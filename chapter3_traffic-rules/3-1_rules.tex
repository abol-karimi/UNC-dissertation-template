\subsection{Traffic Rules}
In this section,
we model the logical form of
the traffic rules.
In the fragment of handbook that we analyzed,
we identified three rules:
Rule \ref{rule:yieldToInside},
\ref{rule:FIFO} and \ref{rule:yieldToRight}.
Here we present the formula for each rule
using the predicates that were introduced in the analysis.
%------------------------
\subsubsection{Rule (\ref{rule:yieldToInside})}
\begin{center}
    $ mustYieldToForRule(V1, V2, yieldToInside) \leftrightarrow $ \\
    $ \Big( atTheIntersection(V1) \; \land $\\
    $ inTheIntersection(V2) \Big) $

\end{center}
In our model of the yield obligation,
we keep track of who must yield to whom,
and an identifier for the corresponding rule
(e.g. $yieldToInside$).
The identifier accounts for the fact that
the mapping from the yield obligations to the actions
depends on the rule corresponding to the obligation.
This will become clear in the definition of
$mustStopToYield$.

%------------------------
\subsubsection{Rule (\ref{rule:FIFO})}
\begin{center}
    $ mustYieldToForRule(V2, V1, firstInFirstOut) \leftrightarrow $ \\
    $ \Big( atTheIntersection(V1) \; \land $\\
    $ atTheIntersection(V2) \; \land $\\
    $ arrivedEarlierThan(V1, V2) \Big) $
\end{center}

%------------------------
\subsubsection{Rule (\ref{rule:yieldToRight})}
\begin{center}
    $ mustYieldToForRule(V1, V2, yieldToRight) \leftrightarrow $ \\
    $ \Big( atTheIntersection(V1) \; \land $\\
    $ atTheIntersection(V2) \; \land $\\
    $ arrivedSameTime(V1, V2) \; \land $\\
    $ isOnRightOf(V2, V1) \Big) $
\end{center}

%-------------------------
\subsubsection{The yield action}
In the case of an uncontrolled intersection,
there is no explicit definition of the yield action in the driver handbook.
We interpret the \emph{yield} action to be \emph{stopping and waiting for a certain amount of time}.
The duration of wait depends on the rule corresponding to the yield obligation.
For rules $firstInFirstOut$ and $yieldToRight$,
we define the duration to be 
until the other vehicle enters the intersection;
thus the duration is 
the same as of the yield obligation.
However,
for rule $yieldToInside$,
the duration depends on a notion of lane \emph{reservation}.
The predicate $mustStopToYield$ defines the yield action:
\begin{center}
    $ mustStopToYield(V1) \; \leftrightarrow \; \exists V2 \Bigg( $ \\
    $ \exists L \Big(  mustYieldToForRule(V1, V2, yieldToInside) \; \land $\\
    $ requestedLane(V1, L) \; \land \; reservedLane(V2, L) \Big) $\\
    $ \lor $ \\
    $ mustYieldToForRule(V1, V2, firstInFirstOut) $ \\
    $ \lor $ \\
    $ mustYieldToForRule(V1, V2, yieldToRight) $ \Bigg).
\end{center}
