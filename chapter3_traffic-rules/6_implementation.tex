\section{Implementation}
\label{sec:implementation}

In this section we briefly explain how to implement
our model of the rules as an ASP program in Clingo.
The logic programs are made available on the web\footnotemark[\ref{foot:link}].
It is interesting to note that the relevant rules for unprotected intersections can be specified in approximately 150 lines of code.

An ASP program is a collection of rules of the form:
\begin{minted}[fontsize=\footnotesize]{prolog}
h :- p_1, ..., p_n, not p_{n+1}, ..., not p_{n+m}.
\end{minted}
The intended meaning of the above sentence is
$$ p_1 \land \dots \land p_n \land  \neg p_{n+1} \land \dots \land \neg p_{n+m} \rightarrow h $$

%---------------------------------
\subsection{Conjunctive condition}
\subsubsection{Positive conjuncts}
Rule (\ref{rule:yieldToInside}) is straightforwardly implemented as:
\begin{minted}[fontsize=\small]{prolog}
mustYieldToForRule(V1, V2, yieldToInside):-
  atTheIntersection(V1),
  inTheIntersection(V2).
\end{minted}

\subsubsection{Positive and negative conjuncts}
\begin{minted}[fontsize=\small]{prolog}
atTheIntersection(V):-
  arrived(V),
  not entered(V).
\end{minted}

%----------------------------------
\subsection{Existential Quantifier}
If we have an existential quantifier
over variables ($T1$ and $T2$ here)
that only appear in positive conjuncts:

\begin{minted}[fontsize=\small]{prolog}
arrivedEarlierThan(V1, V2):-
  arrivedAtTime(V1, T1),
  arrivedAtTime(V2, T2),
  T1 < T2.
\end{minted}

If the existentially quantified variable is referenced only once,
we do not need to name it.
An underscore represents a nameless variable.
For example,

\begin{minted}[fontsize=\small]{prolog}
arrived(V):-
  arrivedAtForkAtTime(V, _, _).
\end{minted}

%---------------------------------
\subsection{Disjunctive condition}
If the definition of a predicate is
a disjunction of two subformulas,
then the definition can be broken down into two definitions
where each subformula is the definition.

For example,
consider the definition of $mustStopToYield$ for uncontrolled intersections.
The condition can be written in a disjunctive form,
by distributing the existential quantifier over disjunctions:
\begin{center}
    $ mustStopToYield(V1) \; \leftrightarrow \; \Bigg( $ \\
    $ (\exists V2)(\exists L) \Big( mustYieldToForRule(V1, V2, yieldToInside)  $\\
    $ \; \land \; requestedLane(V1, L) \; \land \; reservedLane(V2, L) \Big) $\\
    $ \lor $ \\
    $ \exists V2 \Big( mustYieldToForRule(V1, V2, firstInFirstOut) \Big) $ \\
    $ \lor $ \\
    $ \exists V2 \Big( mustYieldToForRule(V1, V2, yieldToRight) $ \Big) \Bigg).
\end{center}

Then we can implement $mustStopToYield$ 
by the following three rules:
\begin{minted}[fontsize=\small]{prolog}
mustStopToYield(V1):-
  mustYieldToForRule(V1, V2, yieldToInside),
  requestedLane(V1, L),
  reservedLane(V2, L).
  
mustStopToYield(V):-
  mustYieldToForRule(V, _, firstInFirstOut).
  
mustStopToYield(V):-
  mustYieldToForRule(V, _, yieldToRight).
\end{minted}


The traffic objects,
i.e. vehicles, lanes, arrival boxes, entrance boxes and exit boxes,
are implemented as polygon meshes in Unreal Engine.
We use Unreal Engine's built-in collision checking
to determine whether two objects overlap.
This enables an efficient and scalable computation of atomic predicates.
If both objects are static,
we query the collision checking only once at the beginning of a scenario.
Otherwise,
we ask Unreal Engine to generate collision events
when a dynamic object begins or ends overlapping with other traffic objects.
Unreal Engine's game time is used as the global time.
We discretize this time to timestamp the events.