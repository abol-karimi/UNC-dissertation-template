\section{Modeling T-intersections}
\label{sec:modeling_T-intersection}

Most of our modeling for an uncontrolled intersection
can be used for an uncontrolled T-intersection.
The additions and alterations needed are as follows.

%--------------------------------------------------
\subsection{Traffic Rules}

%------------------------
\subsubsection{Rule (\ref{rule:T-intersection})}
\begin{center}
    $ mustYieldToForRule(V1, V2, throughRoadFirst) \leftrightarrow $ \\
    $ \Big( atTheIntersection(V1) \; \land $\\
    $ vehicleOnThroughRoad(V2) \; \land $\\
    $ \neg vehicleOnThroughRoad(V1) \Big). $
\end{center}

%------------------------
\subsubsection{Yield action for traffic on minor road}
\begin{center}
    $ mustStopToYield(V1) \; \leftrightarrow \; (\exists V2)(\exists L1)(\exists L2) \Big( $ \\
    $ mustYieldToForRule(V1, V2, throughRoadFirst) \; \land $\\
    $ requestedLane(V1, L1) \; \land \; requestedLane(V2, L2) \; \land $\\
    $ overlaps(L1, L2) \; \land \; \neg leftTheLane(V2, L1) \Big). $
\end{center}

%------------------------
\subsubsection{Rule (\ref{rule:yieldToInside})}
For T-intersections,
we need a different model of the yield action for this rule.
This is because a vehicle that arrives from the through road
and is not turning,
need not stop for a vehicle in front of it that is 
going the same direction.
We choose to skip presenting this nuance and
refer the reader to the implementation file
provided on the web.\footnote{\label{foot:link}\url{https://tinyurl.com/y4lkrn6g}}

%------------------------------Auxiliary Predicates---------------------------
\subsection{Auxiliary Predicates}

%-----------------------------------
\subsubsection{Traffic on through road}
Recall that we consider a vehicle \emph{on through road}
only if it has reached within a certain distance from the intersection.
This is similar to the arrival event for generic uncontrolled intersections
which is generated when a vehicle reaches to a distance from the intersection.
Therefore,
we reuse the arrival event to determine whether a vehicle is on the through road:
\begin{center}
    $ vehicleOnThroughRoad(V) \; \leftrightarrow  $ \\
    $ (\exists F)(\exists T) \Big( arrivedAtForkAtTime(V, F, T) \; \land  $ \\
    $ forkOnThroughRoad(F) \; \land \; \neg exited(V) \Big). $
\end{center}
where
\begin{center}
    $ forkOnThroughRoad(F) \; \leftrightarrow \; (\exists L)(\exists E) \Big( $ \\
    $ laneFromTo(L, F, E) \; \land \; laneCorrectSignal(L, off)  \Big). $
\end{center}