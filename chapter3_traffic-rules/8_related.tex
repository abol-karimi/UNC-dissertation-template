\section{Related work}
Formalizing and monitoring of traffic rules for autonomous vehicles
is studied by Rizaldi et al.~\cite{Rizaldi.2017}.
They consider highway driving and safe distance,
but we focus on intersections and right-of-way.
Our formalization is similar in using a top-down approach.
First they ``codify'' a rule in a temporal logic sentence using abstract atomic predicates
and then ``concretize'' the atomic predicates by giving them a computable definition.
Computation of the predicates are done by code generated by Isabelle.
However,
we codify the rules in a first-order logic sentence and implement them in an ASP logic program,
and compute the concretized atomic predicates using Unreal Engine.
In~\cite{Rizaldi.2017},
given the trajectories
of all vehicles,
the monitor detects any violations of traffic rules.
However,
we determine the legal action at any point in time.
This gives us a high-level controller.
Furthermore,
we easily get a monitor by comparing the required action
with the action taken.


Safety at intersections in terms of collision freedom
is studied by Hilscher et al. \cite{Hilscher.2016}.
The main difference from our work is that they develop a logic to prove safety of controllers,
whereas we formalize the traffic rules to determine the legal action.
Another difference is that the rules that we formalize in this paper concern right-of-way,
whereas they consider safety in terms of collision freedom.
Our concept of \emph{intersection-lane} is inspired by their paper,
but our modeling and implementation is different:
Our definition readily handles different number of lanes and streets connecting to the intersection and captures the geometry of the intersection more faithfully.
In particular, they partition the intersection area into four disjoint regions called ``crossing segments'' and then define intersection-lanes to be certain subsets of this partition.
Two lanes overlap if they share a segment.
In our approach, each intersection-lane is a curved lane specified by a spline.
Unreal Engine calculates whether two such lanes overlap.

Traffic protocols for safe behavior of autonomous vehicles at intersections
were proposed in~\cite{azimi2011vehicular,hafner2013cooperative}.
Approaches to formally verify such protocols using dynamic logic
were developed in~\cite{loos2011safe}.
However,
these papers focus on the safety predicate (safe separation of vehicles).
Unlike our paper,
these works do not explicitly consider the high level traffic behaviors of vehicles.
Furthermore, such protocols rely on vehicle-to-vehicle (V2V) or vehicle-to-infrastructure (V2I) equipment and do not propose a solution where the agents are a mix of autonomous and human drivers. 


OpenDRIVE \cite{ASAM-OD.2019} and 
OpenSCENARIO \cite{ASAM-OSC.2019}
are standards from Association for Standardization of Automation and Measuring Systems (ASAM)
for describing road networks and traffic scenarios
in a tool-independent format.
CARLA partially supports OpenSCENARIO.
Traffic sequence charts,
presented in~\cite{Damm.2018},
is a visual specification language for capturing scenarios,
that is easy to understand and reason.
Another project that
provides schematics for describing traffic scenarios is Open Autonomous Safety from Voyage \cite{OAS.2018}.
Extracting formal specification from natural language automatically has been proposed in ARSENAL~\cite{ghosh2014automatic} and VARED~\cite{badger2014vared}.
However,
these tools only extract specification as LTL formulas~\cite{pnueli1977temporal}.
In contrast,
we manually translate the rules in the driver handbook to first-order formulas
and do not restrict ourselves to temporal logic.
