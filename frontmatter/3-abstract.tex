%The word "Abstract" should be centered 2? below the top of the page. 
%Skip one line, then center your name followed by the title of the 
%thesis/dissertation. Use as many lines as necessary. Centered below the 
%title include the phrase, in parentheses, "(Under the direction of  
%_________)" and include the name(s) of the dissertation advisor(s).
%Skip one line and begin the content of the abstract. It should be 
%double-spaced and conform to margin guidelines. An abstract should not 
%exceed 150 words for a thesis and 350 words for a dissertation. The 
%latter is a requirement of both the Graduate School and UMI's 
%Dissertation Abstracts International.
%Because your dissertation abstract will be published, please prepare and 
%proofread it carefully. Print all symbols and foreign words clearly and 
%accurately to avoid errors or delays. Make sure that the title given at 
%the top of the abstract has the same wording as the title shown on your 
%title page. Avoid mathematical formulas, diagrams, and other 
%illustrative materials, and only offer the briefest possible description 
%of your thesis/dissertation and a concise summary of its conclusions. Do 
%not include lengthy explanations and opinions.
%The abstract should bear the lower case Roman number ii (if you did not 
%include a copyright page) or iii (if you include a copyright page).

\begin{center}
\vspace*{52pt}
% Don't use \Large
{\textbf{ABSTRACT}}
\vspace{11pt}

\begin{singlespace}
    Abolfazl Karimi: Towards Certification of Autonomous Cars: formulating the requirements and testing the compliance\\
(Under the direction of Parasara Sridhar Duggirala)
\end{singlespace}
\end{center}

% Flow:
% - context and problem
% - briefly highlight contributions


%---------------------------------------------------------

To successfully integrate autonomous vehicles as a mode of transportation, we must test these systems against their end-to-end requirements.
%
When AVs and humans (pedestrians, bicyclists, human-driven vehicles) share the road, they must follow the same traffic rules.
%
Using the common traffic rules (written for human drivers) as an engineering requirement poses a challenge due to the ambiguity of the natural language.
%
On the other hand, an AV is fundamentally different from a human, and a simple road test is not sufficient to assess an AV's compliance and skills.
%
This calls for automated, systematic and scalable testing techniques.
%
The focus of this dissertation is on simulation-based testing of the traffic-rules requirements.


%--- Contributions
This dissertation develops techniques for systematic exploration of the test-case space of autonomous vehicles based on two crucial concepts: \emph{complexity} and \emph{coverage}.
%
Here, these concepts are formalized with respect to the traffic rules requirements.
%
The efficiency of finding bugs is improved by incrementally increasing the complexity of a test-case, namely making it harder for an AV to pass a test-case.
%
On the other hand, the diversity of a test-suite is improved by guiding the test-case generation towards increasing the coverage of a test-suite.
%
The framework for formalization of the traffic rules is made more amenable to vetting by the authorities and regulators by narrowing the gap between human intuition and machine language.
%
This is achieved by formalizing traffic rules in first-order logic (FOL) which allows modeling objects and predicates.
\clearpage



% The gap between human intuition and machine language is narrowed by formalizing traffic rules in first-order logic (FOL) which allows modeling objects and predicates.